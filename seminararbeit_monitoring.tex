\documentclass[12pt,a4paper]{report}
\usepackage[latin1]{inputenc}
\usepackage{amsmath}
\usepackage{amsfonts}
\usepackage{amssymb}
\usepackage{makeidx}
\usepackage{graphicx}
\usepackage[left=4.00cm, right=2.00cm, top=4.00cm, bottom=2.00cm]{geometry}
%Tiefe der Nummerierung �ndern, Subsubsection ist Ebene 3
\setcounter{secnumdepth}{3}
\author{Markus �sterle}
\title{Servermonitoring in Rechenzentren}
\begin{document}
	\bibliography{seminararbeit_monitoring}
	\maketitle
	\tableofcontents
	\chapter{Einleitung}
	\section[Warum?]{Warum Monitoring?}
	
	\chapter[Vergleich]{Vergleich mehrerer Monitoring Systeme}
	\section[Performance]{Performance Monitoring}
	\subsection[SCOM]{Microsoft SCOM}
	\subsection[Nagios]{Nagios und Nagios Derivate}
	\subsubsection[OMD]{Open Monitoring Distribution und Check\_MK}
	\subsubsection[Nagios]{Nagios}
	\subsubsection[Shinken]{Shinken}
	\subsection[CA Monitoring]{Computer Associates Monitoring Systeme}
	\section[Logfile]{Logfile Monitoring}
	\subsection[Splunk]{Splunk Monitoring}
	\subsection[ELK]{Elasticsearch - Logstash - Kibana}
	\chapter[Praxis]{Praktische Vorstellung eines Monitoring Systems}
\end{document}