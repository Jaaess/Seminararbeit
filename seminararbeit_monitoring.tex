\documentclass[12pt,a4paper,parskip]{scrreprt}
\usepackage[english,german,ngerman]{babel}
\usepackage[utf8]{inputenc}
\usepackage[T1]{fontenc}
\usepackage{amsmath}
\usepackage{amsfonts}
\usepackage{amssymb}
%http://www.namsu.de/Extra/pakete/Setspace.html
\usepackage{setspace}
\usepackage{makeidx}
\makeindex
\usepackage{graphicx}
%Für Quellcodeschnippsel
\usepackage{listings}
%Überflüssig, dank Dokumentenklasse scrreprt
%\usepackage[left=4.00cm, right=2.00cm, top=4.00cm, bottom=2.00cm]{geometry}
%Glossar anschalten und in TOC aufnehmen und erzeugen
%https://en.wikibooks.org/wiki/LaTeX/Glossary
\usepackage{hyperref} %Klickbare Einträge im Glossar
%\usepackage[toc,acronym]{glossaries}
%\makeglossaries
%\usepackage[nomain,acronym,xindy,toc]{glossaries} % nomain, if you define glossaries in a file, and you use \include{INP-00-glossary}
%http://tex.stackexchange.com/questions/111192/problems-with-xindy-and-glossaries
%\usepackage[nonumberlist,nomain,acronym,toc]{glossaries}
% Acronym muss an erster Stelle stehen, sonst funktioniert die Trennung von Abkürzungsverzeichnis und Glossar nicht
\usepackage[acronym,nonumberlist,toc]{glossaries}
%Glossareinträge
\newglossaryentry{GPLv2}
	{
 	 name=GPLv2,
 	 description={is a programmable machine that receives input,
               stores and manipulates data, and provides
               output in a useful format}
	}
\newglossaryentry{psu}
	{
	name=PSU,
	description={Power Supply Unit, Netzteil},
	plural=PSUs
	}
\newglossaryentry{sla}
	{
	name=SLA,
	description={Service Level Agreement}
	}
%Glossareinträge Ende
%Acronyme
\newacronym{rz}{RZ}{Rechenzentrum}
\newacronym{ram}{RAM}{Random Access Memory}
\newacronym{ca}{CA}{Computer Associates}
\newacronym{fc}{FC}{Fibre Channel}
\newacronym{elk}{ELK}{Elasticsearch - Logstash - Kibana}
\newacronym{it}{IT}{Informationstechnik}
\newacronym{omd}{OMD}{Open Monitoring Distribution}
\newacronym{os}{OS}{Betriebssystem}
\newacronym{etc}{etc.}{etcetera}
\newacronym{itil}{ITIL}{Information Technology Infrastructure Library}
\newacronym{sla}{SLA}{Service Level Agreement}
\newacronym{snmp}{SNMP}{Simple Network Management Protocol}
\newacronym{dv}{DV}{Datenverarbeitung}
\newacronym{bspw}{bspw.}{beispielsweise}
\newacronym{zb}{z.B.}{zum Beispiel}
\newacronym{nrpe}{NRPE}{Nagios Remote Plugin Executor}
\newacronym{san}{SAN}{Storage Area Network}
\newacronym{psu}{PSU}{Power Supply Unit}
\newacronym{rest}{REST}{Representational State Transfer}
\newacronym{api}{API}{Application Programming Interface}
\newacronym{cpu}{CPU}{Central Processing Unit}
\newacronym{itsm}{ITSM}{IT Service Management}
\newacronym{llc}{LLC}{Limited Liability Company}
\newacronym{obs}{OBS}{Open Build Service}
\makeglossaries
%\usepackage{imakeidx}
%Tiefe der Nummerierung ändern, Subsubsection ist Ebene 3
%\setcounter{secnumdepth}{3}
% Bibliothek & Zitierstil
\usepackage[style=authortitle-icomp,backend=biber]{biblatex}
\usepackage[babel,german=guillemets]{csquotes}
%\bibliography{seminararbeit_monitoring.bib} %U.u. deprecated
\addbibresource{seminararbeit_monitoring.bib}
\setuptoc{toc}{bibliography=totoc}
% Abkürzungsverzeichnis mit Nomencl
% siehe http://texwelt.de/wissen/fragen/4942/wie-kann-ich-die-erstellung-eines-abkurzungsverzeichnisses-mit-dem-paket-nomencl-automatisieren
% und http://golatex.de/abkuerzungsverzeichnis-mit-texmaker-t8301.html
% Zum aktualisieren muss "Makeindex" aufgerufen werden, Konfiguration siehe letzter Link
%\usepackage[intoc]{nomencl}
% Befehl umbenennen in abk
%\let\abk\nomenclature
% Überschrift
%\renewcommand{\nomname}{Abkürzungsverzeichnis}
% Punkte zw. Abkürzung und Erklärung
%\setlength{\nomlabelwidth}{.20\hsize}
%\renewcommand{\nomlabel}[1]{#1 \dotfill}
% Zeilenabstnde verkleinern
%\setlength{\nomitemsep}{-\parsep}
%\makenomenclature
% Ende Abkürzungsverzeichnis
\author{Markus Österle}
\title{Monitoring in Rechenzentren am Beispiel von Linux Systemen}
\date{} % setzt ein leeres Datum und löscht quasi das per default verwendete \today von der Titelseite
\begin{document}
	\maketitle
	\setcounter{tocdepth}{3}
	\tableofcontents
	% Gibt das Latex Logo aus...Stelle? 
	%\LaTeXe{}
%Ab hier 1.5facher Zeilenabstand 
	\onehalfspacing
	\chapter{Einleitung}
Das Thema des Servermonitorings hat sich im professionellen \acrshort{rz} Umfeld in den letzten Jahren zu einem großen Thema entwickelt. Durch die steigende Komplexität der Systeme steigen auch die Anforderungen an die Überwachung der selbigen, reichte es in den 90er Jahren noch aus von einem zentralen Rechner (gemeint ist hier ein normaler Personal Computer) per Ping zu Überwachen ob Rechner verfügbar sind, so wird heute ein sehr viel feineres Monitoring erwartet. Nicht nur die Erreichbarkeit der Rechner soll überwacht und statistisch aufbereitet werden, sondern auch die Auslastung und die Gesundheit einzelner Bauteile (bspw. \gls{psu}, \acrshort{ram} Bausteine). \\

In der vorliegenden Arbeit wird das Thema und die Historie des Servermonitorings unter Begrenzung auf das Teilgebiet Linux-Systeme betrachtet. Dies hat zum einen den Grund, dass der Autor viele Jahre Erfahrung mit Linux und Unix Betriebssystemen vorweisen kann und schon einige Erfahrung mit der Überwachung von *nix-Servern gemacht hat.
	\section{Warum Monitoring im Rechenzentrum?}
	
		Überwachung der Serverauslastung
		
		Proaktive Erkennung von (Hardware)Defekten
		
		\gls{sla} Überwachung
		\subsection{Warum Linux?}
	
	\chapter{Vom Ping zum proaktiven Monitoring}
	In den folgenden Unterkapiteln werde ich mich mit der grundlegenden Definiton von Begriffen rund um das Thema Monitoring/Serverüberwachung beschäftigen und einen kleinen Überblick über die Historie geben.
	\section{Historie}
	große Maschinen die alle Software gebündelt hatten und einfacher zu überwachen waren
	Zersplitterung der Software auf viele kleinere Maschinen, dadurch erhöhter Überwachungsaufwand, zusärtlich auch gestiegene Anforderungen an Verfügbarkeit und damit auch an eben dieses Monitoring
	\subsection{ITIL die Quelle allen Übels?}
	\section{Definition Monitoring}
	\subsection{Allgemein}
	\subsection{proaktives Monitoring}
	\subsection{reaktives Monitoring}
	\subsection{Verfügbarkeitsmonitoring}
	\subsection{SLA Monitoring}
	\subsection{Fazit}
	Um alle in den vergangenen Unterkapiteln beschriebenen Anforderungen zu erfüllen ist es notwendig mehrere Überwachungstechnologien zu verbinden, um ein optimales Monitoring zu erhalten. So ist es zur proaktiven Erkennung von Fehlern zum einen unerlässlich die Einträge in Logfiles auswerten zu können, um Fehler zu bemerken lange bevor Sie beginnen sich auf den Produktivbetrieb auszuwirken (bspw. Schreibfehler auf ein per \acrlong{fc} (\acrshort{fc}) angebundenes \gls{san}-System oder einfach I/O Fehler auf eine der Systemfestplatten) und zum anderen auch die Performance der Systeme im Auge zu behalten um sicher zu stellen, dass die Dimensionierung der Systeme richtig ist und nicht plötzliche Engpässe auftreten können.
	\chapter{Monitoring Systeme für Linux Server}
	In diesem Kapitel geht es darum, einen kleinen Überblick über die am Markt verfügbaren Monitoringsysteme zu geben,der Einfachheit halber, wird an dieser Stelle jeweils ein kostenpflichtiges Tool und ein kostenfreies Tool vorgestellt. Dieser Überblick erhebt keinen Anspruch in irgendeiner Art und Weise vollständig zu sein.
	\section{Performance \& SLA Monitoring}
	\subsection{Nagios}
	\subsubsection{Historie}
	Nagios -> Kritik Weiterentwicklung
	Icinga
	Fertige Lösung im Bundle mit mehr Tools
	OMD -> checkmk Änderung im Subskriptionsmodell
	check_mk Versionen Freie Version, Subskriptionen: Innovation, Stable, etc.
	
	\subsubsection{Lizenzmodell}
	Erläuterung GPLv2
	
	\gls{GPLv2}
	\subsubsection{Funktionsumfang}
	
	\subsection{Computer Associates Monitoring Systeme}
	
	\gls{ca}
	
	\subsubsection{Historie}
	\subsubsection{Lizenzmodell}
	\subsubsection{Funktionsumfang}
	\section{Logfile Monitoring}
	\subsection{Splunk Monitoring}
	\subsection{Elasticsearch - Logstash - Kibana}
	\subsection{Syslog-ng und Rsyslog}
	Problematisch in nicht homogenen Umgebungen mit verschiedenen Versionen aktueller Distributionen, weil systemd...ELK Stack ist hier besser weil vielseitiger
	\footcite{systemd2015}
	\chapter{Praktische Vorstellung der Funktionalität eines Monitoring Systems am Beispiel einer OMD Instanz}
	\section{Versuchsaufbau}
	\subsection{verwendete Software}
	
	\subsection{Konfiguration}
	\section{Performancemonitoring}
	\subsection{Überwachung eines entfernten Servers}
	\subsection{Überwachung einer Datenbank}
	\section{Service Level Agreement Monitoring}
	\subsection{Überwachung der Verfügbarkeit einer Datenbank}
	\subsection{Überwachung der Verfügbarkeit einer Webseite}
	% Abkürzungsverzeichnis auf einer neuen Seite ausgeben
	%\printnomenclature
	% Literaturverzeichnis ausgeben und als Eintrag gleichwertig mit einem Kapitel ins Inhaltsverzeichnis eintragen
	\begingroup
	\nocite{*} %Alle Einträge des Literaturverzeichnisses auch ohne Zitierung ausgeben
	\printbibliography 
	\addchaptertocentry{}{Literatur}
	%\addcontentsline{toc}{chapter}{Literatur}
	
	\endgroup
	\printglossary[title=Abkürzungsverzeichnis, type=\acronymtype] % prints just the list of acronyms	
	\printglossary[title=Glossar]
	%\printglossaries	
\end{document}