\documentclass[12pt,a4paper]{report}
\usepackage[latin1]{inputenc}
\usepackage{amsmath}
\usepackage{amsfonts}
\usepackage{amssymb}
\usepackage{makeidx}
\usepackage{graphicx}
\usepackage[left=4.00cm, right=2.00cm, top=4.00cm, bottom=2.00cm]{geometry}
%Tiefe der Nummerierung �ndern, Subsubsection ist Ebene 3
\setcounter{secnumdepth}{3}
\author{Markus �sterle}
\title{Servermonitoring in Rechenzentren am Beispiel von Linux Systemen}
\begin{document}
	\bibliography{seminararbeit_monitoring}
	\maketitle
	\tableofcontents
	\chapter{Einleitung}
	Das Thema des Servermonitorings hat sich im professionellen RZ Umfeld in den letzten Jahren zu einem gro�en Thema entwickelt. Durch die steigende KOmplexit�t der Systeme steigen auch die Anforderungen an die �berwachung der selbigen, reichte es in den 90er Jahren noch aus per Ping zu �berwachen ob Rechner verf�gbar sind, so wird heute ein sehr viel feineres Monitoring erwartet, nicht nur die Erreichbarkeit der Rechner soll �berwacht und statistisch aufbereitet werden, sondern auch die Auslastung und die Gesundheit einzelner Bauteile (bspw. Netzteile, RAM Bausteine).
	\section[Warum?]{Warum Monitoring?}
		�berwachung der Serverauslastung
		
		Proaktive Erkennung von (Hardware)Defekten
		
		SLA �berwachung
		
		
	\chapter[Vergleich]{Vergleich mehrerer Monitoring Systeme}
	\section[Performance]{Performance Monitoring}
	\subsection[SCOM]{Microsoft SCOM}
	\subsection[Nagios]{Nagios und Nagios Derivate}
	\subsubsection[OMD]{Open Monitoring Distribution und Check\_MK}
	\subsubsection[Nagios]{Nagios}
	\subsubsection[Shinken]{Shinken}
	\subsection[CA Monitoring]{Computer Associates Monitoring Systeme}
	\section[Logfile]{Logfile Monitoring}
	\subsection[Splunk]{Splunk Monitoring}
	\subsection[ELK]{Elasticsearch - Logstash - Kibana}
	\chapter[Praxis]{Praktische Vorstellung eines Monitoring Systems}
\end{document}