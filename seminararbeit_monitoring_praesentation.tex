\documentclass[11pt]{beamer}
\usetheme{Madrid}
\usepackage[utf8]{inputenc}
\usepackage[english,german]{babel}
\usepackage[T1]{fontenc}
\usepackage{amsmath}
\usepackage{amsfonts}
\usepackage{amssymb}
\usepackage{graphicx}
\author{Markus Österle}
\title{Servermonitoring im Rechenzentrum}
%\setbeamercovered{transparent} 
%\setbeamertemplate{navigation symbols}{} 
%\logo{} 
%\institute{} 
%\date{} 
%\subject{} 
\begin{document}

\begin{frame}
\titlepage
\end{frame}

%\begin{frame}
%\tableofcontents
%\end{frame}

\begin{frame}{Übersicht - roter Faden}
\begin{itemize}
	\item Monitoring was ist das eigentlich? - Definitionen
	\item Warum Monitoring?
	\item Wie wird es gemacht? - verwendete Software
\end{itemize}
\end{frame}
\begin{frame}{Definitionen}
\begin{itemize}
	\item Monitoring
	\item proaktives Monitoring
	\item reaktives Monitoring
	\item SLA Monitoring
\end{itemize}
\end{frame}
\begin{frame}{Monitoring}
	Laut Duden: \\
	
	\glqq [Dauer]beobachtung (eines bestehenden Systems)\grqq
\end{frame}
\begin{frame}{Proaktives Monitoring}
	content...
\end{frame}
\begin{frame}{Reaktives Monitoring}
	content...
\end{frame}
\begin{frame}{SLA Monitoring}
	Definition SLA \\
	Definition SLA Monitoring
\end{frame}
\begin{frame}{Nagios}

\end{frame}
\begin{frame}{ELK}

\end{frame}
\end{document}