\documentclass[12pt,a4paper]{report}
\usepackage[utf8]{inputenc}
\usepackage[german]{babel}
% Verwendete Schriftarten
\renewcommand*\sfdefault{ugq}
\fontfamily{uaq}\selectfont
\usepackage[T1]{fontenc}
\usepackage{amsmath}
\usepackage{amsfonts}
\usepackage{amssymb}
\usepackage{graphicx}
\usepackage{kpfonts}
% Bibliothek & Zitierstil
\usepackage[style=authortitle-icomp,backend=biber]{biblatex}
\usepackage[babel,german=guillemets]{csquotes}
% Formatierung
\usepackage[a4paper, margin=2cm, left=4cm, top=4cm, headheight=0pt, headsep=0pt]{geometry}
% \usepackage{showframe} % blendet Seitenränder ein

% Setzt den Zeilenabstand auf echte 1.5 Zeilen
\usepackage[onehalfspacing]{setspace}

% Paket um das Layout grafisch darstellen zu können
\usepackage{layout}

% Abkürzungsverzeichnis mit Nomencl
% siehe http://texwelt.de/wissen/fragen/4942/wie-kann-ich-die-erstellung-eines-abkurzungsverzeichnisses-mit-dem-paket-nomencl-automatisieren
% und http://golatex.de/abkuerzungsverzeichnis-mit-texmaker-t8301.html
% Zum aktualisieren muss "Makeindex" aufgerufen werden, Konfiguration siehe letzter Link
\usepackage[intoc]{nomencl}
% Befehl umbenennen in abk
\let\abk\nomenclature
% Überschrift
\renewcommand{\nomname}{Abkürzungsverzeichnis}
% Punkte zw. Abkürzung und Erklärung
\setlength{\nomlabelwidth}{.20\hsize}
\renewcommand{\nomlabel}[1]{#1 \dotfill}
% Zeilenabstände verkleinern
\setlength{\nomitemsep}{-\parsep}
\makenomenclature
% Ende Abkürzungsverzeichnis
\bibliography{Seminararbeit.bib} 
\author{Markus Österle \\
Matrikelnummer: 20131041}
\date{} % setzt ein leeres Datum und löscht quasi das per default verwendete \today von der Titelseite
\title{Seminararbeit \\ im Studiengang \\ Verwaltungsinformatik \\ Thema: \linebreak Internet der Dinge - Gefahr oder Chance?}
%Beginn des eigentlichen Dokuments
\begin{document}
\maketitle
\tableofcontents
\newpage
\chapter{Einleitung}
Das Internet of Things (IoT) \abk{IoT}{Internet of Things (Deutsche Übersetzung: Internet der Dinge)} ist zur Zeit in aller Munde. Sei es bei der 

\cite{vanDam2015}
\footcite{vanDam2015}

\chapter{Definition}
Lt. Wikipedia \footcite{wikiIoTen} handelt es sich beim Internet der Dinge \\
Lt. Wikipedia \footcite{wikiIoTde} \\

Test \cite{ta1} \footnote{\cite{ta1}}
\newpage
Test2 \cite{ta2} \footcite{ta1}
test123

\chapter{Zielsetzung dieser Arbeit}
\section{Grobe Zielsetzung}

Standard


\textsf{Standard}
\abk{z.B.}{zum Beispiel}
\abk{etc.}{etcetera}
\abk{z.B.}{zum Beispiel}


% Erzeugt eine Seite die das Layout bildlich darstellt,\newpage erzeugt vorher einen Seitenumbruch
\newpage
\layout
\newpage
% Abkürzungsverzeichnis ausgeben
\printnomenclature

% Literaturverzeichnis ausgeben und als Eintrag gleichwertig mit einem Kapitel ins Inhaltsverzeichnis eintragen
\begingroup
\printbibliography 
\addcontentsline{toc}{chapter}{Literatur}
\endgroup

\end{document}

